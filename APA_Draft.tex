% Options for packages loaded elsewhere
\PassOptionsToPackage{unicode}{hyperref}
\PassOptionsToPackage{hyphens}{url}
%
\documentclass[
  english,
  man]{apa6}
\usepackage{amsmath,amssymb}
\usepackage{lmodern}
\usepackage{ifxetex,ifluatex}
\ifnum 0\ifxetex 1\fi\ifluatex 1\fi=0 % if pdftex
  \usepackage[T1]{fontenc}
  \usepackage[utf8]{inputenc}
  \usepackage{textcomp} % provide euro and other symbols
\else % if luatex or xetex
  \usepackage{unicode-math}
  \defaultfontfeatures{Scale=MatchLowercase}
  \defaultfontfeatures[\rmfamily]{Ligatures=TeX,Scale=1}
\fi
% Use upquote if available, for straight quotes in verbatim environments
\IfFileExists{upquote.sty}{\usepackage{upquote}}{}
\IfFileExists{microtype.sty}{% use microtype if available
  \usepackage[]{microtype}
  \UseMicrotypeSet[protrusion]{basicmath} % disable protrusion for tt fonts
}{}
\makeatletter
\@ifundefined{KOMAClassName}{% if non-KOMA class
  \IfFileExists{parskip.sty}{%
    \usepackage{parskip}
  }{% else
    \setlength{\parindent}{0pt}
    \setlength{\parskip}{6pt plus 2pt minus 1pt}}
}{% if KOMA class
  \KOMAoptions{parskip=half}}
\makeatother
\usepackage{xcolor}
\IfFileExists{xurl.sty}{\usepackage{xurl}}{} % add URL line breaks if available
\IfFileExists{bookmark.sty}{\usepackage{bookmark}}{\usepackage{hyperref}}
\hypersetup{
  pdftitle={Neural correlates of choosing between pairs of description and experience-based risky options},
  pdfauthor={A. Zeynep Enkavi1, Gabriela Tavares2, \& Antonio Rangel1},
  pdflang={en-EN},
  pdfkeywords={risk, uncertainty, lottery, reinforcement learning},
  hidelinks,
  pdfcreator={LaTeX via pandoc}}
\urlstyle{same} % disable monospaced font for URLs
\usepackage{graphicx}
\makeatletter
\def\maxwidth{\ifdim\Gin@nat@width>\linewidth\linewidth\else\Gin@nat@width\fi}
\def\maxheight{\ifdim\Gin@nat@height>\textheight\textheight\else\Gin@nat@height\fi}
\makeatother
% Scale images if necessary, so that they will not overflow the page
% margins by default, and it is still possible to overwrite the defaults
% using explicit options in \includegraphics[width, height, ...]{}
\setkeys{Gin}{width=\maxwidth,height=\maxheight,keepaspectratio}
% Set default figure placement to htbp
\makeatletter
\def\fps@figure{htbp}
\makeatother
\setlength{\emergencystretch}{3em} % prevent overfull lines
\providecommand{\tightlist}{%
  \setlength{\itemsep}{0pt}\setlength{\parskip}{0pt}}
\setcounter{secnumdepth}{-\maxdimen} % remove section numbering
% Make \paragraph and \subparagraph free-standing
\ifx\paragraph\undefined\else
  \let\oldparagraph\paragraph
  \renewcommand{\paragraph}[1]{\oldparagraph{#1}\mbox{}}
\fi
\ifx\subparagraph\undefined\else
  \let\oldsubparagraph\subparagraph
  \renewcommand{\subparagraph}[1]{\oldsubparagraph{#1}\mbox{}}
\fi
% Manuscript styling
\usepackage{upgreek}
\captionsetup{font=singlespacing,justification=justified}

% Table formatting
\usepackage{longtable}
\usepackage{lscape}
% \usepackage[counterclockwise]{rotating}   % Landscape page setup for large tables
\usepackage{multirow}		% Table styling
\usepackage{tabularx}		% Control Column width
\usepackage[flushleft]{threeparttable}	% Allows for three part tables with a specified notes section
\usepackage{threeparttablex}            % Lets threeparttable work with longtable

% Create new environments so endfloat can handle them
% \newenvironment{ltable}
%   {\begin{landscape}\centering\begin{threeparttable}}
%   {\end{threeparttable}\end{landscape}}
\newenvironment{lltable}{\begin{landscape}\centering\begin{ThreePartTable}}{\end{ThreePartTable}\end{landscape}}

% Enables adjusting longtable caption width to table width
% Solution found at http://golatex.de/longtable-mit-caption-so-breit-wie-die-tabelle-t15767.html
\makeatletter
\newcommand\LastLTentrywidth{1em}
\newlength\longtablewidth
\setlength{\longtablewidth}{1in}
\newcommand{\getlongtablewidth}{\begingroup \ifcsname LT@\roman{LT@tables}\endcsname \global\longtablewidth=0pt \renewcommand{\LT@entry}[2]{\global\advance\longtablewidth by ##2\relax\gdef\LastLTentrywidth{##2}}\@nameuse{LT@\roman{LT@tables}} \fi \endgroup}

% \setlength{\parindent}{0.5in}
% \setlength{\parskip}{0pt plus 0pt minus 0pt}

% \usepackage{etoolbox}
\makeatletter
\patchcmd{\HyOrg@maketitle}
  {\section{\normalfont\normalsize\abstractname}}
  {\section*{\normalfont\normalsize\abstractname}}
  {}{\typeout{Failed to patch abstract.}}
\patchcmd{\HyOrg@maketitle}
  {\section{\protect\normalfont{\@title}}}
  {\section*{\protect\normalfont{\@title}}}
  {}{\typeout{Failed to patch title.}}
\makeatother
\shorttitle{Description versus experience-based risk}
\keywords{risk, uncertainty, lottery, reinforcement learning\newline\indent Word count: X}
\DeclareDelayedFloatFlavor{ThreePartTable}{table}
\DeclareDelayedFloatFlavor{lltable}{table}
\DeclareDelayedFloatFlavor*{longtable}{table}
\makeatletter
\renewcommand{\efloat@iwrite}[1]{\immediate\expandafter\protected@write\csname efloat@post#1\endcsname{}}
\makeatother
\usepackage{lineno}

\linenumbers
\usepackage{csquotes}
\ifxetex
  % Load polyglossia as late as possible: uses bidi with RTL langages (e.g. Hebrew, Arabic)
  \usepackage{polyglossia}
  \setmainlanguage[]{english}
\else
  \usepackage[main=english]{babel}
% get rid of language-specific shorthands (see #6817):
\let\LanguageShortHands\languageshorthands
\def\languageshorthands#1{}
\fi
\ifluatex
  \usepackage{selnolig}  % disable illegal ligatures
\fi
\newlength{\cslhangindent}
\setlength{\cslhangindent}{1.5em}
\newlength{\csllabelwidth}
\setlength{\csllabelwidth}{3em}
\newenvironment{CSLReferences}[2] % #1 hanging-ident, #2 entry spacing
 {% don't indent paragraphs
  \setlength{\parindent}{0pt}
  % turn on hanging indent if param 1 is 1
  \ifodd #1 \everypar{\setlength{\hangindent}{\cslhangindent}}\ignorespaces\fi
  % set entry spacing
  \ifnum #2 > 0
  \setlength{\parskip}{#2\baselineskip}
  \fi
 }%
 {}
\usepackage{calc}
\newcommand{\CSLBlock}[1]{#1\hfill\break}
\newcommand{\CSLLeftMargin}[1]{\parbox[t]{\csllabelwidth}{#1}}
\newcommand{\CSLRightInline}[1]{\parbox[t]{\linewidth - \csllabelwidth}{#1}\break}
\newcommand{\CSLIndent}[1]{\hspace{\cslhangindent}#1}

\title{Neural correlates of choosing between pairs of description and experience-based risky options}
\author{A. Zeynep Enkavi\textsuperscript{1}, Gabriela Tavares\textsuperscript{2}, \& Antonio Rangel\textsuperscript{1}}
\date{}


\authornote{

The authors made the following contributions. A. Zeynep Enkavi: Formal Analysis, Writing - Original Draft Preparation, Writing - Review \& Editing; Gabriela Tavares: Conceptualization, Investigation; Antonio Rangel: Conceptualization, Writing - Original Draft Preparation, Writing - Review \& Editing.

Correspondence concerning this article should be addressed to A. Zeynep Enkavi, 1200 E. Colorado Blvd., Pasadena, CA 91125. E-mail: \href{mailto:zenkavi@caltech.edu}{\nolinkurl{zenkavi@caltech.edu}}

}

\affiliation{\vspace{0.5cm}\textsuperscript{1} Division of Humanities and Social Sciences, California Institute of Technology\\\textsuperscript{2} Google}

\abstract{
Enter abstract here
}



\begin{document}
\maketitle

\hypertarget{introduction}{%
\section{Introduction}\label{introduction}}

\hypertarget{materials-and-methods}{%
\section{Materials and Methods}\label{materials-and-methods}}

\hypertarget{participants}{%
\subsection{Participants}\label{participants}}

27 Caltech undergraduates were recruited for the experiment. Two subjects could not be scanned due to discomfort in the scanner so data from 25 subjects (10 female, mean age = 21) are analyzed. Each subjects completed 300 trials broken down to five runs completed in a single sessions. Participants were compensated \$30 for their time and received additional earnings based on their performance as described below. All procedures were approved by the Caltech's IRB and all subjects provided informed consent prior to participation.

\hypertarget{task}{%
\subsection{Task}\label{task}}

Participants completed the task in the fMRI scanner. The task consisted of 300 trials broken down into five runs of 60 trials. Each trial began with a central fixation cross that remained on the screen for a random inter-trial interval (ITI) between four and seven seconds plus a variable amount depending on the participant's response time as described below.

In each trial, the participant had to choose between two pairs presented on the left and right of the screen separated with a gray vertical line. Each pair contained a fractal at the top and a lottery at the bottom. The fractals were randomized across participants from a sample of 25 but they remained the same for the duration of the experiment for a given participant. Each fractal was associated with a probability of a \texttt{\$}1 reward, which were not shown on the screen. These probabilities drifted slowly and independently between 0.25 and 0.75. The initial value for each probability was sampled uniformly within the bounds. After every trial the change to each probability was sampled from a Gaussian distribution centered at 0 and with \(\sigma = 0.025\). If the sampled change value pushed a probability out of bounds, its sign was reversed.

Participants were told that reward probabilities of the fractals drifted slowly and independently but were not informed about the bounds or the drift rate. Instead they learned about each fractal's likelihood to yield a reward by observing the outcomes in each trial as described below.

In addition to the fractals each pair involved a lottery. The lottery for the left option was presented as a pie chart representing its reward probability with the amount of reward listed below this. The right lottery was fixed for all trials and depicted with the text ``REF.'' The varying left lottery was drawn from the following set of twenty probability and reward combinations, each occuring three times per run: \{(1, \texttt{\$}0.50), (0.25, \texttt{\$}2), (0.2, \texttt{\$}2.50), (0.1, \texttt{\$}5), (1, \texttt{\$}0.10), (0.1, \texttt{\$}1), (0.05, \texttt{\$}2), (0.01, \texttt{\$}10), (1, \texttt{\$}0.30), (0.3, \texttt{\$}1), (0.15, \texttt{\$}2), (0.1, \texttt{\$}3), (1, \texttt{\$}0.70), (0.7, \texttt{\$}1), (0.35, \texttt{\$}2), (0.1, \texttt{\$}7), (1, \texttt{\$}0.90), (0.9, \texttt{\$}1), (0.45, \texttt{\$}2), (0.1, \texttt{\$}9)\}. The fixed right lottery consisted of a 50\% probability of winning \texttt{\$}1.

Participants had up to four seconds to respond in each trial. All choices were reported using the right hand and pressing the ``2'' key for the left and the ``3'' key for the right pair. If a choice was not indicated within this time participants saw a message that read ``No response recorded!'' and the experiment continued to the next trial. Trials without a response did not add any reward to the final payout. Trials where a response was recorded faster than four seconds, the remaining time was added to the ITI.

Following the fixation cross and before viewing the pairs of fractals and gambles participants saw two percentages on the screen. Stacked vertically, the top percentage indicated the chance that the reward from that trial would be drawn from the fractals and the bottom percentage the chance that the reward would be drawn from the lotteries. These percentages ranged from zero to one hundred at intervals of 10\% yielding eleven unique values. Values below 30\% or above 70\% occurred five times per run while other values occurred 6 times.

After each choice two draws determined the reward amount of that trial. The first draw, based on the percentages presented following the fixation cross, determined whether the trial's reward was drawn from the fractal or the lottery of the pair chosen by the participant. Depending on the outcome of this draw the fractal or the lottery of the participant's choice was highlighted with a red circle. The second draw determined the exact reward amount based on the reward probability and amount for the fractal or the lottery of the chosen pair. The final reward amount was shown in the center of the screen in red. Thus, the participants were incentivized both to indicate their true preferences in each trial and to learn about the fractal reward probabilities. Moreover to facilitate the learning of these fractal reward probabilities the outcomes of a draw from both fractals were also shown on each fractal, regardless of whether the trial reward depended on this outcome based on the first draw. This reward screen remained for three seconds.

At the end of the experiment participants received a sum of the rewards from 175 randomly chosen trials added to their base compensation, which was adjusted to be within the bounds of \texttt{\$}80 and \texttt{\$}120.

\hypertarget{behavioral-data-preprocessing}{%
\subsection{Behavioral data preprocessing}\label{behavioral-data-preprocessing}}

\hypertarget{neuroimaging-data-preprocessing}{%
\subsection{Neuroimaging data preprocessing}\label{neuroimaging-data-preprocessing}}

\hypertarget{data-analysis-software}{%
\subsection{Data analysis software}\label{data-analysis-software}}

The list of software packages and their versions used for analyses can be found at \ldots{}

\hypertarget{results}{%
\section{Results}\label{results}}

\hypertarget{raw-behavioral-results}{%
\subsection{Raw behavioral results}\label{raw-behavioral-results}}

\hypertarget{cognitive-modeling}{%
\subsection{Cognitive modeling}\label{cognitive-modeling}}

\hypertarget{neuroimaging-results}{%
\subsection{Neuroimaging results}\label{neuroimaging-results}}

\hypertarget{discussion}{%
\section{Discussion}\label{discussion}}

\newpage

\hypertarget{references}{%
\section{References}\label{references}}

\begingroup
\setlength{\parindent}{-0.5in}
\setlength{\leftskip}{0.5in}

\hypertarget{refs}{}
\begin{CSLReferences}{0}{0}
\end{CSLReferences}

\endgroup


\end{document}
